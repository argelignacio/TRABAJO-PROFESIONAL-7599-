\documentclass{article}

\usepackage{arxiv}

\usepackage[utf8]{inputenc} % allow utf-8 input
\usepackage[T1]{fontenc}    % use 8-bit T1 fonts
\usepackage{hyperref}       % hyperlinks
\usepackage{url}            % simple URL typesetting
\usepackage{booktabs}       % professional-quality tables
\usepackage{amsfonts}       % blackboard math symbols
\usepackage{nicefrac}       % compact symbols for 1/2, etc.
\usepackage{microtype}      % microtypography
\usepackage{cleveref}       % smart cross-referencing
\usepackage{lipsum}         % Can be removed after putting your text content
\usepackage{graphicx}
\usepackage{natbib}
\usepackage{doi}
\usepackage[english]{babel}

\usepackage{graphicx}


\title{Análisis comparativo de comunidades en Ethereum: de PoW a PoS}

\newif\ifuniqueAffiliation
% Comment to use multiple affiliations variant of author block 
\uniqueAffiliationtrue
\ifuniqueAffiliation % Standard variant of author block
\author{{
	\hspace{1mm}Ignacio Argel} \\
	FIUBA\\
	\texttt{iargel@fi.uba.ar} \\
	%% examples of more authors
	\And
	{\hspace{1mm}Manuel V.~Battan} \\
	FIUBA\\
	\texttt{mvbattan@fi.uba.ar} \\
	\AND
	{\hspace{1mm}Lucas Bilo} \\
	FIUBA\\
	\texttt{lbilo@fi.uba.ar} \\
	\And
	{\hspace{1mm}Nahuel H.~Spiguelman} \\
	FIUBA\\
	\texttt{nspiguelman@fi.uba.ar} \\
}


\begin{document}

\maketitle

\begin{abstract}
    El cambio de Proof of Work a Proof of Stake en la blockchain descentralizada de Ethereum tiene un impacto directo en la función desempeñada por los mineros, ya que son reemplazados por Stakers como nuevos validadores de la red. La disparidad en las dinámicas de este nuevo protocolo plantea interrogantes sobre si ha ocasionado modificaciones en el comportamiento de los usuarios que operan en él, ya sean stakers o participantes frecuentes de la red.
    Las investigaciones previas en Ethereum han priorizado aspectos técnicos, dejando un vacío en el análisis de interacciones entre cuentas y la formación de comunidades. Esto es especialmente notorio en la ausencia de estudios que aborden cómo el comportamiento de usuarios y comunidades ha cambiado con la migración de PoW a PoS.
    Para este análisis utilizaremos un modelo de Deep Learning cuya finalidad es crear embeddings de nodos que representen, de una manera inteligente, las relaciones entre addresses en la red. A su vez, el método propuesto es superador en términos de limitaciones computacionales y de utilización de features respecto a algoritmos tradicionales para analizar grafos.

\end{abstract}
\clearpage

\section{Introducción}

La blockchain Ethereum ha sido una de las innovaciones más significativas en el mundo de la tecnología financiera en la última década. Desde su lanzamiento en 2015, Ethereum ha operado bajo un protocolo de consenso conocido como Proof of Work (PoW), el cual ha demostrado ser robusto y confiable en la validación de transacciones y la seguridad de la red, pero presentando problemas como el alto consumo de energía y la centralización de la minería de bloques. Debido a estos problemas, Ethereum ha migrado hacia el protocolo de consenso conocido como Proof of Stake (PoS), el cual incluye varios cambios en la manera de garantizar la integridad durante las transacciones, mejorando la escalabilidad y la eficiencia de la red.
Algunas preguntas significativas que surgen a partir del cambio de paradigma: ¿Existe un cambio en el comportamiento de los usuarios debido a esta transición? ¿Cuáles son los comportamientos invariantes o que siguen siendo vigentes después del cambio? ¿Cómo variaron las relaciones entre usuarios y comunidades?

El presente proyecto tiene como objetivo principal contestar estas preguntas. A través de un análisis exhaustivo, buscamos proporcionar una visión clara de los patrones de comportamiento de quienes operan sobre la red de Ethereum en sus dos protocolos de consenso principales, PoW y PoS. Al hacerlo, esperamos contribuir a la comprensión de las implicaciones de esta transición para la comunidad Blockchain, los desarrolladores y los usuarios de Ethereum.


\subsection{Investigación}

\subsubsection{Artículos relacionados}

\paragraph{Ethereum - Yellow paper } \mbox{}\\
Para establecer una base sólida y comprender exhaustivamente los aspectos fundamentales de Ethereum, incluyendo su red y el proceso de transacciones, optamos por examinar el documento original de Ethereum. \newline

\href{https://ethereum.github.io/yellowpaper/paper.pdf}{ETHEREUM: A SECURE DECENTRALISED GENERALISED TRANSACTION LEDGER}

El documento proporciona una riqueza de información crucial, detallando temas fundamentales como los campos de las transacciones, la naturaleza de los bloques, el protocolo de consenso (entonces basado en Prueba de Trabajo, PoW), y otros aspectos de suma importancia.

\paragraph{Artículos muy relacionados}\mbox{}\\

Como una sugerencia inicial y como punto de partida para abordar el tema, se nos recomendó revisar el siguiente documento académico acerca de la detección de comunidades ilícitas en la red de Bitcoin. Aunque en sus conclusiones se menciona la falta de resultados satisfactorios, el documento proporciona una definición de comunidades, propone diversas técnicas y evalúa el rendimiento mediante métricas específicas.
Una idea inicial era encarar el trabajo sobre esto mismo, pero en ethereum, relacionándolo con cambio de protocolo de consenso.
\newline

\href{https://www.mdpi.com/1099-4300/25/7/1069}{Illegal Community Detection in Bitcoin Transaction Networks}
\newline

Luego de eso, buscamos artículos recientes que recopilaran algo de la situación actual científica respecto a la red de ethereum y su análisis como un grafo. Este paper nos dio un punto de partida en lo que respecta a la diversidad de tipos de cuentas y tipos de transacciones dentro de la red. Viendo esto, nos dimos cuenta que debíamos limitar el scope del grafo, ya que los distintos tipos de cuenta y transacciones se comportan de una manera muy diferente. Debido a esto, decidimos focalizarnos en las transacciones que tienen un value, transaccionado el token Ether.
\newline

\href{https://homes.cs.aau.dk/~Arijit/Papers/IEEE_Blockchain_2022_Ethereum_Blockchain_Graph_Data_Survey.pdf}{Graph Analysis of the Ethereum Blockchain Data:A Survey of Datasets, Methods, and Future Work}
\newline

El último paper y estrechamente vinculado que hemos identificado se enfoca en la identificación de comunidades dentro de redes sociales. En este contexto, se sugiere la aplicación de la matriz de covarianza de salida de baja dimensionalidad sobre los datos de transacciones, seguido por la extracción de los K autovectores. Posteriormente, se emplea el algoritmo de K-means para la partición de los nodos en K grupos distintos. Es interesante que la conclusión y las pruebas definitivas, se hacen sobre un grafo de tamaño reducido. 

\href{https://arxiv.org/pdf/2101.06406}{Community Detection in Blockchain Social Networks}
\newline

Una hipótesis surgida de estos análisis es si es la mejor opción buscar comunidades utilizando K-means, ya que el algoritmo obligatoriamente separa en K clusters, sin dejar a nadie fuera de una comunidad. 

\subsubsection{Papers complementarios}
Si bien estos papers no fueron foco principal de nuestra investigación, los revisamos de forma rápida viendo si encontrábamos insights inspiradores o novedosos.

Este paper es uno de los primeros trabajos científicos que definen una comunidad, lo que la hace una lectura muy interesante.

\href{https://www.ncbi.nlm.nih.gov/pmc/articles/PMC122977/pdf/pq1202007821.pdf}{Community structure in social and
biological networks}
\newline

En el siguiente se presenta el primer estudio empírico a gran escala sobre el uso de Inline Assembly en contratos inteligentes de Ethereum, analizando millones de contratos desde diversas perspectivas técnicas. Proporciona nuevas observaciones y percepciones valiosas para el desarrollo de contratos inteligentes y la evolución de Solidity y sus compiladores.

\href{https://drive.google.com/file/d/111l8_vf7Gt7RxygjuBM8WQWLTQL700jY/view}{Large-Scale Empirical Study of Inline Assembly on 7.6 Million Ethereum Smart Contracts}
\newline

******INSERTE SUS PAPERS LEIDOS ACA*******

\subsection{Materiales y métodos}
\subsubsection{Fuente de datos}

La primera complicación con la que nos enfrentamos fue la búsqueda de una fuente de datos íntegra, sin limitaciones significativas en términos de cantidad y accesibilidad. Inicialmente, exploramos la posibilidad de utilizar APIs con el propósito de obtener muestras extensas y almacenarlas. Sin embargo, la restricción en el número de solicitudes permitidas, junto con la complejidad inherente de realizar una cantidad considerable de peticiones y la gestión de almacenamiento para tal volumen de datos, nos llevó a desechar esta estrategia.

Ejemplos de APIs revisadas (entre otras):
\begin{list}{\labelitemi}{\leftmargin=4em}
    \item \href{https://www.blockchain.com/explorer/api/blockchain_api}{Blockchain Data} 
    \item \href{https://etherscan.io/apis}{Etherscan}
\end{list}

Posteriormente, consideramos la posibilidad de replicar un nodo. Contábamos con diversas alternativas, si bien replicar un nodo plantea ciertas dificultades técnicas y de almacenamiento. Las alternativas que evaluamos fueron las siguientes:
\begin{itemize}
    \item Nodo completo (Full node): Este tipo de nodo conserva y administra una réplica integral de la cadena de bloques de Ethereum, aunque con ciertos recortes (pruning). Esto abarca todas las transacciones efectuadas en la red, así como el estado actualizado de todas las cuentas y contratos inteligentes hasta un punto en el tiempo determinado. La implementación de este tipo de nodo requiere una capacidad de almacenamiento de aproximadamente 1.5 TB.
    \item Nodo de archivo (Archive node): Se trata de una categoría de nodo que preserva todo el historial completo de la cadena de bloques desde su bloque inicial hasta el más reciente. Esto incluye todas las transacciones, contratos inteligentes, eventos y estados históricos de todas las cuentas. La variante más ligera de este tipo de nodo requiere alrededor de 10 TB de almacenamiento hasta mayo de 2022.
\end{itemize}
Sin embargo, esta opción fue desechada debido a la limitación de recursos de almacenamiento disponibles y a la falta de necesidad de poseer los datos en su totalidad.

Por último, hemos identificado un  \href{https://cloud.google.com/blog/products/data-analytics/ethereum-bigquery-public-dataset-smart-contract-analytics}{dataset en BigQuery} que, aunque no está sincronizado en tiempo real, se actualiza diariamente. Las consultas realizadas en este conjunto de datos son gratuitas y, para propósitos prácticos, su límite es inalcanzable. Los campos disponibles en este conjunto de datos son los siguientes: 
\begin{itemize}
    \item \textbf{block\_hash}: El identificador único de un bloque en la blockchain, el cual es generado a partir de la información contenida en el bloque mediante una función de hash.
    \item \textbf{block\_number}: El número de secuencia del bloque dentro de la blockchain.
    \item \textbf{block\_timestamp}: Es la hora y fecha exacta en la que el bloque fue minado.
    \item \textbf{transaction\_hash}: El identificador único de una transacción específica, el cual es generado a partir de la información de la transacción mediante una función de hash.
    \item \textbf{transaction\_index}: La posición de la transacción dentro del bloque. 
    \item \textbf{nonce}: Un número utilizado para garantizar que cada transacción sólo pueda ser procesada una vez. Es específico de la cuenta del remitente y se incrementa con cada transacción enviada desde esa cuenta.
    \item \textbf{from\_address}: La dirección de Ethereum del remitente de la transacción.
    \item \textbf{to\_address}: La dirección de Ethereum del destinatario de la transacción. Puede ser la dirección de un usuario o la dirección de un smart contract.
    \item \textbf{value}: Cantidad de Ether transferida en la transacción, está expresada en wei.
    \item \textbf{value\_lossless}: Es el mismo valor que el campo “value” pero está expresado de otra forma para evitar cierta precisión en algunas aplicaciones. 
    \item \textbf{gas}: es la tarifa que se paga en la transacción. Está expresada en wei.
    \item \textbf{gas\_price}: Es el precio en Ether de una unidad de gas usada en la transacción. Está expresada en wei.
    \item \textbf{max\_fee\_per\_gas}: es la cantidad máxima de gas que el usuario remitente está dispuesto a pagar para que su transacción se incluya en un bloque. Está expresada en wei.
    \item \textbf{max\_priority\_fee\_per\_gas}: Es determinada por el usuario remitente y es opcional. Indica la cantidad de gas que se está dispuesto a pagar a los mineros como “propina” por la transacción. Está expresada en wei.
    \item \textbf{transaction\_type}: Indica el tipo de la transacción. Puede ser 0 para transacciones regulares o 2 para transacciones EIP-1559.
    \item \textbf{chain\_id}: Identificador único para la cadena de bloques (Ethereum Mainnet, Testnet, entre otros). 
    \item \textbf{r y s}: Son componentes de la firma digital de la transacción. En conjunto se utilizan para verificar que la transacción fue firmada por el propietario de la clave privada correspondiente a la dirección de origen de la transacción.
    \item \textbf{v}: Es un parámetro de la firma digital de la transacción, se usa para la recuperación de la clave pública. 
    \item \textbf{y\_parity}: Es un parámetro de la firma digital de la transacción y es una versión más compacta y moderna de lo que indica el campo v. Se introdujo en las transacciones de tipo EIP-1559 y EIP-2930.
\end{itemize}
A continuación, se adjunta una transacción con datos reales:
\begin{tabbing}
  "block\_hash": "0xbe6dd03c251417b51020f5358c5981fd1627575c256dd98d6d24dd7687997d32",\\
  "block\_number": "17329020", \\
  "block\_timestamp": "2023-05-24 12:23:35.000000 UTC",\\
  "transaction\_hash": "0x5fe28c92492cfd4107b867da66036cf108a2a7b09baed29fdc543d2ba5778574",\\
  "transaction\_index": "79",\\
  "nonce": "4894815",\\
  "from\_address": "0x56eddb7aa87536c09ccc2793473599fd21a8b17f",\\
  "to\_address": "0x86ba5d7600e1ce97fdf821e96a0b4ec729843ab2",\\
  "value": "9208340000000000",\\
  "value\_lossless": "9208340000000000",\\
  "gas": "207128",\\
  "gas\_price": "32342796700",\\
  "input": "0x",\\
  "max\_fee\_per\_gas": "102000000000",\\
  "max\_priority\_fee\_per\_gas": "2000000000",\\
  "transaction\_type": "2",\\
  "chain\_id": "1",\\
  "access\_list": [],\\
  "r": "0x8e7de580875602ab722aaeb890eb2bfb88443e34338fd2936408e89d4e5b456c",\\
  "s": "0x937cd39f66fb771a8c1216b296f1c32b42c5dc013054b808596cad794ca084a",\\
  "v": "0x0",\\
  "y\_parity": null\\

\end{tabbing}


 Con acceso a este conjunto de datos y la capacidad de realizar consultas sobre cualquier ventana temporal desde el inicio de la cadena hasta el día anterior al actual, tenemos la información necesaria para nuestros análisis.

\subsubsection{node2vec}
node2vec es un algoritmo para aprendizaje de representaciones de nodos en redes. Su objetivo es mapear los nodos de un grafo a un espacio vectorial continuo de baja dimensión, preservando las propiedades y estructuras del grafo original. Funciona en dos fases principales:

Generación de Trayectorias Aleatorias: node2vec utiliza una combinación de técnicas de paseo aleatorio (random walks) sesgado para explorar el grafo. Ajusta dos parámetros, 
$p$ y $q$, que permiten controlar el equilibrio entre la exploración de nuevas partes del grafo (similar a un paseo de profundidad primero, DFS) y la explotación de áreas cercanas al nodo inicial (similar a un paseo de amplitud primero, BFS). El parámetro $p$ controla la probabilidad de regresar al nodo anterior, y $q$ ajusta la probabilidad de visitar nodos más alejados.

Optimización de la Representación: Una vez generadas las trayectorias, node2vec aplica técnicas de aprendizaje no supervisado similares a las usadas en el procesamiento de lenguaje natural, como Skip-gram, para optimizar los vectores de los nodos. El objetivo es maximizar la probabilidad de co-ocurrencia de nodos que están cerca en las trayectorias generadas.

De esta manera, node2vec logra capturar tanto la estructura local como global del grafo, produciendo representaciones vectoriales útiles para tareas posteriores como la clasificación de nodos, detección de comunidades y predicción de enlaces.

\subsubsection{NetworkX}
NetworkX es un paquete de Python para la creación, manipulación y estudio de la estructura, dinámica y funciones de redes complejas. Es ampliamente utilizado en investigación y aplicaciones prácticas debido a su versatilidad y facilidad de uso. NetworkX permite trabajar con grafos (redes) de cualquier tamaño y tipo, incluidos grafos dirigidos, no dirigidos y multigrafos. Ofrece una amplia gama de algoritmos para análisis de redes, como medidas de centralidad, detección de comunidades, algoritmos de búsqueda de caminos, y más. Además, se integra bien con otras bibliotecas científicas de Python, como NumPy, SciPy y Matplotlib, facilitando el análisis y visualización de datos de redes.

\subsubsection{UMAP}
UMAP, Uniform Manifold Approximation and Projection, es un algoritmo de reducción de dimensionalidad no lineal que destaca por su eficiencia y habilidad para preservar las estructuras y relaciones locales en datos de alta dimensión. A través de la construcción de un grafo ponderado basado en la conectividad local de los puntos en el espacio de alta dimensión, UMAP busca optimizar una representación en un espacio de menor dimensión que conserve estas relaciones. Su capacidad para capturar estructuras complejas y no lineales, combinada con su eficaz preservación de la estructura global y local de los datos, lo convierte en una herramienta poderosa para visualizar y explorar conjuntos de datos grandes y complejos, haciéndolo especialmente relevante en campos como la ciencia de datos y el aprendizaje automático.

\subsection{Pregunta a responder}
Durante la etapa inicial de revisión de literatura relacionada con la temática, percibimos, por lo expuesto en éstas, una notable carencia de investigación exhaustiva sobre el análisis de comunidades en Ethereum. En general, los estudios más completos y con mejores resultados se han desarrollado predominantemente sobre la blockchain de Bitcoin.

Al inicio de nuestra investigación, surgió la idea de analizar el comportamiento de los usuarios en la red de Ethereum en relación con el cambio de protocolo de consenso implementado en septiembre de 2022.

Como mencionamos previamente, este cambio introduce una diferencia tanto técnica como teórica en el proceso de aprobación de nuevos bloques de transacciones dentro de la red.

Dicha diferencia podría ser un factor motivador en la modificación del comportamiento de los usuarios que utilizan la red diariamente para realizar acuerdos o transferencias de activos.

Por lo tanto, la pregunta principal que se busca responder en este trabajo es: ¿Realmente se produjo un cambio en la composición de las comunidades? ¿Cómo ha impactado este cambio en el uso de la blockchain de Ethereum, considerando que el nuevo protocolo permite (y además fomenta) una mayor participación general de los usuarios en la comunidad?

\subsection{Resultado esperado}
Para responder a las preguntas planteadas, es necesario desarrollar una serie de conceptos que abarquen desde la formulación de nuestro propio concepto de comunidad hasta los métodos para identificarlas.

En este estudio, al igual que en muchos de los trabajos previamente revisados, se considerará a priori como característica principal de la relación entre dos nodos, la cantidad de transacciones entre ellos. Esta relación puede no ser simétrica y, si es necesario, podríamos incluir otros datos transaccionales que aporten valor a la hora de definir la importancia de la misma.

Con base en esta relación, se buscará identificar comunidades en las que las interacciones entre dos o más nodos pertenecientes a la misma comunidad sean más significativas que las interacciones con nodos externos a dicha comunidad.
Cabe mencionar que la definición de comunidades puede ser difusa, dependiendo del concepto inicial de comunidad, del método de definición y de la naturaleza del problema. Además, adherimos a la idea presentada en la literatura revisada, de que un nodo no necesariamente debe pertenecer a una comunidad. Si pertenece, es a una única comunidad, pero podría no estar en ninguna debido a la baja calidad de sus relaciones con otros nodos de la red.

Con esto en mente, se propone analizar cualitativa y cuantitativamente las comunidades detectadas y su composición, en un periodo de tiempo determinado, antes y después de la implementación del cambio al protocolo de consenso Proof of Stake.

Para comprender mejor el problema, es esencial conocer los datos, lo que permitirá formular preguntas más específicas y refinar las hipótesis a medida que se analicen los datos y, consecuentemente, las comunidades.
Finalmente, todo esto contribuirá a construir un método para agrupar distintos nodos en un grafo de transacciones de Ethereum, permitiendo identificar sus comunidades.


\section{Metodología}

\subsection{Análisis exploratorio}

Una vez conseguidos los datos, debíamos empezar a analizar los mismos, con el fin de conocer mejor los datos, su evolución y en base a eso tomar la decisión de que subset de datos pre y post cambio de protocolo de consenso vamos a elegir para la comparación final.

Para esto, decidimos plantear una serie de métricas 



\bibliographystyle{alpha}
%\bibliography{sample}
%\nocite{*}

\end{document}